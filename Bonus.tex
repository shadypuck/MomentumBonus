\documentclass{article}

\usepackage[letterpaper, margin=1in, footskip=0.25in]{geometry}
\usepackage{fancyhdr}
\usepackage{indentfirst}
\usepackage{amsmath}
\usepackage{csquotes}

\MakeOuterQuote{"}
\setcounter{secnumdepth}{0}

\begin{document}


\pagestyle{fancy}
\fancyhf{}
\rfoot{Labalme \thepage}
\renewcommand{\headrulewidth}{0pt}

\noindent Steven Labalme\par
\noindent Mr. Bilak\par
\noindent AP Physics C: Mechanics, 1\par
\noindent 11 February 2019\par

\begin{center}
\section{Momentum Bonus}
\end{center}

\subsection{Task}
Given the following two equations, which describe the conservation of momentum and kinetic energy, respectively, before and after an elastic collison, derive formulas for the final velocity of each object ($v_{1f}$ and $v_{2f}$) in terms of only these object's initial velocities and masses (not each other's final velocities).
\begin{align}
    m_1v_{1i}+m_2v_{2i} &= m_1v_{1f}+m_2v_{2f}\\
    \frac{1}{2}m_1{v_{1i}}^2+\frac{1}{2}m_2{v_{2i}}^2 &= \frac{1}{2}m_1{v_{1f}}^2+\frac{1}{2}m_2{v_{2f}}^2
\end{align}

\subsection{Solution}
Solve first for one of the variables (say, $v_{1f}$) and then for the other variable using an identical procedure ($v_{2f}$, if $v_{1f}$ is solved for first).\par
Treat equations 1 and 2 as a two-variable system of equations, where the two variables are $v_{1f}$ and $v_{2f}$. Solve the "system" by substitution i.e. solve one equation (say, equation 1) for $v_{2f}$ and plug the result into equation 2. This will create a third equation (equation 3) without $v_{2f}$. Finally, solve equation 3 algebraically for $v_{1f}$ to give the first of the two formulas (equation 4). This procedure can be carried out as follows.

\begin{align*}
    m_1v_{1i}+m_2v_{2i} &= m_1v_{1f}+m_2v_{2f}\\
    m_1v_{1i}+m_2v_{2i}-m_1v_{1f} &= m_2v_{2f}\\
    v_{2f} &= \frac{m_1v_{1i}+m_2v_{2i}-m_1v_{1f}}{m_2}
\end{align*}


\end{document}