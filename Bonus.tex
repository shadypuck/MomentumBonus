\documentclass{article}

\usepackage[letterpaper, margin=1in, footskip=0.25in]{geometry}
\usepackage{fancyhdr}
\usepackage{indentfirst}
\usepackage{amsmath}
\usepackage{csquotes}

\MakeOuterQuote{"}
\setcounter{secnumdepth}{0}

\begin{document}


\pagestyle{fancy}
\fancyhf{}
\rfoot{Labalme \thepage}
\renewcommand{\headrulewidth}{0pt}

\noindent Steven Labalme\par
\noindent Mr. Bilak\par
\noindent AP Physics C: Mechanics, 1\par
\noindent 11 February 2019\par

\begin{center}
\section{Momentum Bonus}
\end{center}

\subsection{Task}
Given the following two equations, which describe the conservation of momentum and kinetic energy, respectively, before and after an elastic collison, derive formulas for the final velocity of each object ($v_{1f}$ and $v_{2f}$) in terms of only these object's initial velocities and masses (not each other's final velocities).
\begin{align}
    m_1v_{1i}+m_2v_{2i} &= m_1v_{1f}+m_2v_{2f}\\
    \frac{1}{2}m_1{v_{1i}}^2+\frac{1}{2}m_2{v_{2i}}^2 &= \frac{1}{2}m_1{v_{1f}}^2+\frac{1}{2}m_2{v_{2f}}^2
\end{align}

\subsection{Solution}
Solve first for one of the variables (say, $v_{1f}$). Then, since the decision to designate one object, object 1, and the other, object 2, is arbitrary, flip the numbers in the solution for $v_{1f}$ to find the solution for $v_{2f}$.\par
To solve for $v_{1f}$, first, manipulate equations 1 and 2 to determine a simpler relation (equation 3) between the velocities. This will quicken the substitution and simplification described later.
\vspace{0em}
\begin{align*}
    m_1v_{1i}+m_2v_{2i} &= m_1v_{1f}+m_2v_{2f}\\
    m_1v_{1i}-m_1v_{1f} &= m_2v_{2f}-m_2v_{2i}\\
    m_1(v_{1i}-v_{1f}) &= m_2(v_{2f}-v_{2i})
\end{align*}
\begin{align*}
    \frac{1}{2}m_1{v_{1i}}^2+\frac{1}{2}m_2{v_{2i}}^2 &= \frac{1}{2}m_1{v_{1f}}^2+\frac{1}{2}m_2{v_{2f}}^2\\
    m_1{v_{1i}}^2+m_2{v_{2i}}^2 &= m_1{v_{1f}}^2+m_2{v_{2f}}^2\\
    m_1{v_{1i}}^2-m_1{v_{1f}}^2 &= m_2{v_{2f}}^2-m_2{v_{2i}}^2\\
    m_1({v_{1i}}^2-{v_{1f}}^2) &= m_2({v_{2f}}^2-{v_{2i}}^2)\\
    m_1(v_{1i}-v_{1f})(v_{1i}+v_{1f}) &= m_2(v_{2f}-v_{2i})(v_{2f}+v_{2i})
\end{align*}
\vspace{0em}
\begin{equation*}
    \frac{m_1(v_{1i}-v_{1f})(v_{1i}+v_{1f}) = m_2(v_{2f}-v_{2i})(v_{2f}+v_{2i})}{m_1(v_{1i}-v_{1f}) = m_2(v_{2f}-v_{2i})}\\
\end{equation*}
\vspace{-1.8em}
\begin{align*}
    v_{1i}+v_{1f} &= v_{2f}+v_{2i}\\
    v_{1i}-v_{2i} &= v_{2f}-v_{1f}\tag{3}
\end{align*}

Treat equations 1 and 3 as a two-variable system of equations, where the two variables are $v_{1f}$ and $v_{2f}$. Solve the "system" by substitution i.e. solve one equation (say, equation 3) for $v_{2f}$ and plug the result into equation 1. This will create an equation without $v_{2f}$.
\vspace{0em}
\begin{align*}
    v_{1i}-v_{2i} &= v_{2f}-v_{1f}\\
    v_{2f} &= v_{1i}-v_{2i}+v_{1f}
\end{align*}
\vspace{-0.5em}
\begin{equation*}
    m_1v_{1i}+m_2v_{2i} = m_1v_{1f}+m_2(v_{1i}-v_{2i}+v_{1f})
\end{equation*}

\newpage
Finally, solve this equation algebraically for $v_{1f}$ to give the first of the two formulas (equation 4).
\vspace{0em}
\begin{align*}
    m_1v_{1i}+m_2v_{2i} &= m_1v_{1f}+m_2v_{1i}-m_2v_{2i}+m_2v_{1f}\\
    m_1v_{1i}+2m_2v_{2i}-m_2v_{1i} &= (m_1+m_2)v_{1f}\\
    v_{1f} &= \frac{m_1v_{1i}+2m_2v_{2i}-m_2v_{1i}}{m_1+m_2}\tag{4}
\end{align*}

Flip the subscripts in equation 4 to give the solution to $v_{2f}$ (equation 5).

\begin{equation*}
    v_{2f} = \frac{m_2v_{2i}+2m_1v_{1i}-m_1v_{2i}}{m_1+m_2}\tag{5}
\end{equation*}

To restate, the final solutions for $v_{1f}$ and $v_{2f}$ (equations 4 and 5) are listed below.

\begin{equation*}
    \fbox{$v_{1f}$ = $\dfrac{m_1v_{1i}+2m_2v_{2i}-m_2v_{1i}}{m_1+m_2}$} \hspace{0.76in} \fbox{$v_{2f}$ = $\dfrac{m_2v_{2i}+2m_1v_{1i}-m_1v_{2i}}{m_1+m_2}$}
\end{equation*}

\end{document}